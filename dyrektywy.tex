\documentclass[a4paper,12pt]{article}
\usepackage[polish]{babel}
\usepackage{graphicx}
\usepackage[utf8]{inputenc}
\usepackage[T1]{fontenc}
\usepackage{babel}
\usepackage{gensymb}
\usepackage{hyperref}
\usepackage{subfig}
\usepackage{graphicx}
\usepackage{float}
\usepackage{amsmath}
\usepackage{amssymb}
\usepackage{geometry}
\geometry{margin=2.5cm}


\title{Dyrektywy poprawności sylogizmu}
\author{Jakub Figura}
\date{23 Listopada 2025}

\begin{document}

\maketitle

\begin{itemize}
    \item Dyrektywa wstępna: niechaj nie będzie czwartego terminu!
    \item Dyrektywa 1: termin średni powinien być przynajmniej w jednej z przesłanek wzięty w całym zakresie („rozłożony''). Termin jest rozłożony, jeżeli występuje jako: \textbf{X}a..., \textbf{X}e..., ...e\textbf{X}, albo ...o\textbf{X}
    \item Dyrektywa 2: przynajmniej jedna przesłanka powinna być twierdząca (czyli zdanie a lub i)
    \item Dyrektywa 3: przynajmniej jedna przesłanka powinna być ogólna (czyli zdanie a lub e)
    \item Dyrektywa 4: wniosek zawsze i tylko wtedy jest przeczący, gdy jedna przesłanka jest przecząca; wniosek jest twierdzący wtedy i tylko wtedy, gdy obie przesłanki są twierdzące
    \item Dyrektywa 5: jeżeli jedna przesłanka jest szczegółowa, to wniosek jest szczegółowy; zaś jeżeli wniosek jest szczegółowy, to obie przesłanki muszą być ogólne.
    \item Dyrektywa 6: jeżeli termin jest „rozłożony'' we wniosku, to \textbf{musi} być „rozłożony'' także w przesłance
    \item Dyrektywa 7: jeżeli brakującą przesłanką może być zdanie szczegółowe lub ogólne, to naley wybrać zdanie szczegółowe - \textbf{dotyczy tylko zadań z entymematów}
\end{itemize}

\end{document}


