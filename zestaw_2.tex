\documentclass[a4paper,12pt]{article}
\usepackage[polish]{babel}
\usepackage{graphicx}
\usepackage[utf8]{inputenc}
\usepackage[T1]{fontenc}
\usepackage{babel}
\usepackage{gensymb}
\usepackage{hyperref}
\usepackage{subfig}
\usepackage{graphicx}
\usepackage{float}
\usepackage{amsmath}
\usepackage{amssymb}
\usepackage{geometry}
\geometry{margin=2.5cm}

\title{Zestaw 2}
\author{Jakub Figura}
\date{19 października 2025}

\begin{document}

\maketitle

\section*{Zadanie 1}
Sprawdź metodą zerojedynkową skróconą, czy następująca formuła jest tautologią:
\begin{enumerate}
    \item $((p\rightarrow q)\wedge p)\rightarrow q$
    \item $(p \rightarrow q) \rightarrow (\neg q \rightarrow \neg p)$
    \item $(\neg p \vee q)\rightarrow (p\rightarrow q)$
    \item $\neg(p \vee q) \rightarrow (\neg p \wedge \neg q)$
    \item $\neg(p \wedge q) \rightarrow (\neg p \vee \neg q)$
    \item $((p\rightarrow q) \wedge(q\rightarrow r)) \rightarrow (p \rightarrow q)$
    \item $((p \rightarrow q) \wedge(r\rightarrow p))\rightarrow (p \vee q)$
    \item $\neg(p\  \bot\  q) \rightarrow(p\  /\  q)$
    \item $((p\ / q) \downarrow (r\rightarrow\neg p))\rightarrow(r\wedge q)$
    \item $\neg((\neg p \downarrow q)\rightarrow(r \vee s))\rightarrow ((q \wedge s)\vee \neg p)$
    \item $((r \wedge \neg q)\rightarrow(\neg r \equiv p))\rightarrow((r / \neg q)\vee (\neg r \downarrow p))$
    \item $(((p / q ) \equiv r) \wedge ((r \downarrow q)\  / \ p))\rightarrow (((r\equiv p )\wedge q)\rightarrow p)$
    \item $\neg(\neg(\neg p\ / \neg q)\rightarrow\neg(q \downarrow r))\rightarrow\neg(\neg(r\wedge p)\vee \neg q)$
    \item $((p \downarrow q) \wedge \neg(\neg r \vee q))\rightarrow \neg((p \vee q)\bot r)$
    \item $((p \equiv q) \rightarrow ( r\downarrow q)) \rightarrow (\neg(q\ /\ r)\rightarrow p )$    
\end{enumerate}
Uwaga! Jeżeli formuła nie jest tautologią podaj dla jakich wartości jest fałszywa. 
\newpage
\section*{Zadanie 2}
Sprawdź metodą zerojedynkową skróconą, czy następująca formuła jest tautologią:
\begin{enumerate}
    \item $((p \ \bot \ q)\ /\ r)\rightarrow((r\equiv \neg q)\rightarrow\neg p)$
    \item $((p \vee q)\ / \ (\neg p \rightarrow q))\rightarrow ((\neg p \rightarrow \neg q) \wedge (p \rightarrow \neg r))$
    \item $((p \ / \neg q ) \wedge\neg(r\wedge \neg p))\rightarrow ((\neg q \vee r) \ / \neg (p \bot \neg r))$
    \item $((p \vee \neg q)\bot (r \downarrow s))\rightarrow \neg ((q \rightarrow p) \equiv \neg (r \vee s))$
    \item $\neg(\neg(p \downarrow\neg q)\rightarrow(r\equiv p))\rightarrow ((\neg p \vee r) \rightarrow(q \downarrow r))$
\end{enumerate}
\section*{Zadanie dodatkowe}
Adam, Bartek i Czarek są podejrzani o uchylanie się od płacenia podatków. Składają następujące zeznania pod przysięgą:\\
\\
\textbf{Adam}: Bartek jest winny, a Czarek nie jest winny.\\ 
\textbf{Bartek}: Jeżeli Adam jest winny, to winny jest też Czarek.\\
\textbf{Czarek}: Nie jestem winny, ale przynajmniej jedna osoba z posotałych jest winna.\\
\\
Przyjmując, że zdanie p - oznacza „Adam jest winny'', q - „Bartek jest winny'', r - „Czarek jest winny'' odpowiedz na pytanie:\\
Zakładając, że zeznania każdego są prawdziwe, kto jest winny?\footnote{Zadanie pochodzi z H. de Swart, \textit{Philosophical and Mathematical Logic}, Springer 2018, s. 60 - 61.} 
\end{document}
