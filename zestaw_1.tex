\documentclass[a4paper,12pt]{article}
\usepackage[polish]{babel}
\usepackage{graphicx}
\usepackage[utf8]{inputenc}
\usepackage[T1]{fontenc}
\usepackage{babel}
\usepackage{gensymb}
\usepackage{hyperref}
\usepackage{subfig}
\usepackage{graphicx}
\usepackage{float}
\usepackage{amsmath}
\usepackage{amssymb}
\usepackage{geometry}
\geometry{margin=2.5cm}
\usepackage[backend=biber]{biblatex}
\addbibresource{e3.bib}

\title{Zestaw 1}
\author{Jakub Figura}
\date{Październik 2025}

\begin{document}

\maketitle

\subsection*{Zadanie 1} Wskaż zdania w sensie logiki:
\begin{enumerate}
    \item \textit{Helicobacter pylori} jest spiralną bakterią, kolonizującą błonę śluzową żołądka.
    \item Idź do dziekanatu.
    \item Kraków jest stolicą Polski.
    \item Logika jest bardzo łatwa.
    \item Jeżeli Szymon jest starszy od Jakuba, to Jakub jest młodszy od Szymona. 
    \item Czym jest nauka? 
    \item Iloczyn dwóch liczb ujemnych jest liczbą dodatnią. 
    \item Korzystanie ze sztucznej inteligencji powinno być zabronione. 
    \item Konstytucja RP weszła w życie 17 października 1997 roku.
    \item Łukasz lubi kawę lub Łukasz lubi herbatę.
\end{enumerate}
\subsection*{Zadanie 2} Wskaż, które z podanych wyrażeń KRZ są formułami:
\begin{enumerate}
    \item $\neg p /q$
    \item $prq\bot qrp$
    \item $\neg(p\bot q)\equiv (\neg r \downarrow s)$
    \item $\neg(\neg p \vee q)\wedge\neg(p\rightarrow r)$
    \item $(\rightarrow)\wedge(\bot)\equiv(\neg p)$
    \item $(p\vee q)\wedge \rightarrow q $
    \item $\neg p\  \neg q $
    \item $((p \rightarrow q)\wedge p \rightarrow q $
\end{enumerate}
\subsection*{Zadanie 3} Przetłumacz poniższe zdania na język KRZ. 
\begin{enumerate}
    \item Jeżeli Jan jest adwokatem, to Jan jest prawnikiem. 
    \item W przerwie między zajęciami zjem ciastko lub wypiję kawę.
    \item Jan jest prokuratorem lub Jan nie jest prokuratorem. 
    \item Szymon jest sędzią lub Szymon jest posłem. 
    \item Tylko jeżeli liczba jest podzielna przez 3, to jest podzielna przez 9. 
    \item Zdam egzamin z logiki, o ile będę się uczył. 
    \item Karol nie lubi matematyki lub Karol nie lubi fizyki.
    \item Nieprawda, że Karol pije kawę lub pisze książkę.
    \item Nieprawda, że Karol nie pije kawy lub nie pisze książki. 
    \item Nieprawda, że ani Karol nie pije kawy ani Karol nie pisze książki.
    \item Nieprawda, że Jakub konfiguruje sieci komputerowe i zna architekturę komputera. 
    \item Albo tylko jeżeli Karol lubi logikę, to Karol jest prawnikiem albo Karol jest filozofem. 
    \item Jakub programuje wtedy i tylko wtedy, gdy ma dużo czasu wolnego, o ile nie musi sprzątać mieszkania i wychodzić na spacer z psem. 
    \item Adam czyta filozofię tylko jeżeli ma czas wolny oraz tworzy artykuły naukowe zawsze i tylko wtedy, gdy odszukał inspiracje w dziełach filozofów. 
\end{enumerate}
\subsection*{Zadanie 4} Zadania tekstowe
\begin{enumerate}
    \item Jeżeli Jan jest komornikiem, to ukończył studia prawnicze. Ale Jan nie ukończył studiów prawniczych. A więc Jan nie jest komornikiem. 
    \item Jeżeli stan wojenny obowiązuje i Sejm nie zebrał się na posiedzenie, to Prezydent na wniosek Rady Ministrów wydaje rozporządzenie z mocą ustawy. Stan wojenny obowiązuje, jeżeli doszło do zbrojnej napaści na terytorium RP lub istnieje zewnętrzne zagrożenie państwa. Zatem, jeżeli nie doszło do zbrojnej napaści na terytorium RP i nie istnieje zewnętrzne zagrożenie państwa, to jeżeli Prezydent na wniosek Rady Ministrów nie wydaje rozporządzenia z mocą ustawy, to Sejm zebrał się na posiedzenie. 
    \item Jeżeli odzież własna pracownika uległa zniszczeniu lub znacznemu zabrudzeniu, to pracodawca dostarczy pracownikowi nieodpłatnie odzież. A zatem, jeżeli pracodawca nie dostarcza pracownikowi nieodpłatnie odzieży, to nieprawda, że odzież własna pracownika uległa zniszczeniu i nieprawda, że odzież własna pracownika uległa zabrudzeniu. 
    \item Jeżeli Karol jest komunistą, to jest członkiem Międzynarodówki. Nieprawda, że Karol strajkuje z robotnikami lub domaga się uspołecznienia środków produkcji. Zatem nieprawda, że Karol jest komunistą lub jest członkiem Międzynarodówki.
    \item Tylko jeżeli liczba jest podzielna przez 3, to jest podzielna przez 9. Liczba jest podzielna przez 9. Zatem liczba jest podzielna przez 3. 
    \item Albo będę się pilnie uczył, albo nie zdam egzaminu z logiki. Jeżeli egzamin z logiki jest łatwy i nie będę się pilnie uczyć, to zdam egzamin z logiki. Zatem jeżeli egzamin z logiki nie jest łatwy, to jeżeli nie będę się pilnie uczył, to nie zdam egzaminu z logiki. 
    \item Jeżeli Jan jest prokuratorem, to Jan nie jest posłem. Zawsze i tylko wtedy, gdy ani Jan nie jest prokuratorem ani Jan nie jest sędzią, to Jan jest posłem. A zatem jeżeli Jan jest posłem, to Jan nie jest sędzia i Jan nie jest prokuratorem. 
    \item Tylko jeżeli ani nie pada deszcz, ani nie jest pochmurnie, to Ania wybiera się w góry. Nieprawda, że zarazem Ania wybiera się w góry i jest pochmurnie. A zatem jeżeli pada deszcz i jest pochmurnie, to nieprawda, że Ania wybiera się w góry. 
    \item Jeżeli przepis szczególny nie stanowi inaczej, to termin przedawnienia wynosi sześć lat. Nieprawda, że jeżeli roszczenie dotyczy świadczeń okresowych lub roszczeń zwianych z prowadzeniem działalności gospodarczej, to termin przedawnienia wynosi sześciu lat. A zatem, jeżeli roszczenie dotyczy świadczeń okresowych, to termin przedawnienia nie wynosi sześciu lat lub przepis szczególny nie stanowi inaczej. 
    \item Albo matematyka jest odkrywana, albo jest tworzona. Jeżeli równania matematyczne antycypują wyniki eksperymentalne, to matematyka jest odkrywana lub równania są mądrzejsze niż ci, którzy je napisali. A zatem, jeżeli ani równania nie są mądrzejsze niż ci, którzy je napisali, ani równania matematyczne nie antycypują wyników eksperymentalnych, to nieprawda, że matematyka jest tworzona. 
    \item Albo król Francji jest łysy, albo król Francji ma bujną czuprynę. Jeżeli król Francji nosi perukę, to król Francji ma bujną czuprynę. A zatem, jeżeli ani król Francji nie ma bujnej czupryny, ani król Francji nie pielęgnuje fryzury, to nieprawda, że król Francji nosi perukę. 
    \item Zawsze i tylko wtedy, jeżeli nieprawda, że jeżeli nie nasilają się animozje, to w społecznej świadomości pojawia się resentyment, to albo urzeczywistnia się mechanizm ucieczki od wolności, albo nie nasilają się animozje albo żadne z nich. Ani w społecznej świadomości nie pojawia się resentyment, ani nieprawda, że jednostki nie cechuje nonkonformizm. A zatem, jeżeli jednostki nie cechuje nonkonformizm i nasilają się animozje, to urzeczywistnia się mechanizm ucieczki od wolności. 
    \item Albo jeżeli Ola nie odróżnia nocą grabów pospolitych od buków karbowanych, to Ola nie zna ekstensji nazw dendrologicznych, albo ani nieprawda, że Ola nie próżnowała w trakcie lekcji biologii, ani Ola nie odróżnia nocą grabów pospolitych od buków karbowanych. Zawsze i tylko wtedy, jeżeli albo Ola zna ekstensję nazw dendrologicznych, albo Ola próżnowała w trakcie lekcji biologii, to Ola jest ekspertką od wiązów. A zatem, jeżeli Ola jest ekspertką od wiązów i Ola nie próżnowała w trakcie lekcji biologii, to Ola odróżnia nocą garby pospolite od buków karbowanych. 
    \item Jeżeli nieprawda, że ani Kazimierz nie uznaje za wiedzę tylko poznania naukowego, ani Kazimierz nie uznaje przeżycia mistycznego za źródło wartościowej wiedzy, to nieprawda, że zawsze i tylko wtedy, jeżeli Kazimierz jest antyirracjonalistą, to Kazimierz uznaje przeżycie mistyczne za źródło wartościowej wiedzy. Kazimierz uznaje za wiedzę tylko poznanie naukowe. A zatem, jeżeli Kazimierz uznaje przeżycie mistyczne za źródło wartościowej wiedzy, to Kazimierz nie jest antyirracjonalistą.
\end{enumerate}

\end{document}
