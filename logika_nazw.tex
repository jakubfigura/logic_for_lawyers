\documentclass[a4paper,12pt]{article}
\usepackage[polish]{babel}
\usepackage{graphicx}
\usepackage[utf8]{inputenc}
\usepackage[T1]{fontenc}
\usepackage{babel}
\usepackage{gensymb}
\usepackage{hyperref}
\usepackage{subfig}
\usepackage{graphicx}
\usepackage{float}
\usepackage{amsmath}
\usepackage{amssymb}
\usepackage{geometry}
\geometry{margin=2.5cm}% Required for inserting images

\title{Zestaw 4}
\author{Jakub Figura}
\date{21 grudnia 2025}

\begin{document}

\maketitle
\subsection*{1. Wprowadzenie}
\begin{table}[h!]
    \centering
    \title{\textbf{Przekształcenia zdań kategorycznych}}\\
    \begin{tabular}{|c|c|}
    \hline
    \textbf{Zdanie} & \textbf{Formalizacja}\\
    \hline
     Tylko S są P   & PaS\\
    \hline
     Tylko S nie są P  &S'aP\\
    \hline
    Tylko niektóre S są P & SiP$\wedge$SoP\\
    Tylko niektóre S nie są P & SiP$\wedge$SoP\\
    \hline
    \end{tabular}
\end{table}
\begin{table}[h!]
    \centering
    \title{\textbf{Operacja obwersji}}\\
    \begin{tabular}{|c|c|c|c|c|}
    \hline
     \textbf{Zdanie}    & XaY &XeY &XiY &XoY\\
    \hline
     \textbf{Obwersja}  &XeY' &XaY' &XoY' &XiY' \\
    \hline
    \end{tabular}
\end{table}
\begin{table}[h!]
    \centering
    \title{\textbf{Operacja konwersji}}\\
    \begin{tabular}{|c|c|c|c|c|}
    \hline
     \textbf{Zdanie}    & XaY &XeY &XiY &XoY\\
    \hline
     \textbf{Konwersja}  &YiX &YeX &YiX &brak \\
    \hline
    \end{tabular}
\end{table}
Kontrapozycja zupełna = O + K + O
\subsection*{1. Kontrapozycja zupełna}
Przeprowadź kontrapozycję zupełną następujących zdań:
\begin{enumerate}
    \item Każdy kwadrat jest prostokątem.
    \item Żaden płaz nie jest ssakiem.
    \item Niektóre ptaki nie są latające.
    \item Niektóre liczby są pierwsze.
\end{enumerate}
\newpage
\subsection*{2. Kwadrat logiczny}
\begin{table}[h!]
    \centering
    \title{\textbf{Relacje w kwadracie logicznym}}\\
    \begin{tabular}{|c|c|}
    \hline
    \textbf{Typ zdań} & \textbf{Relacja}\\
    \hline
     zdania ogólne  & przeciwieństwo\\
    \hline
     zdania szczegółowe  & podprzeciwieństwo\\
    \hline
    zdania ogólne i odpowiadające im jakością zdania szczegółowe & podporządkowanie\\
    \hline
    zdania ogólne i nieodpowiadające im jakością zdania szczegółowe & sprzeczność\\
    \hline
    \end{tabular}
\end{table}
\begin{enumerate}
Przprowadź wnioskowanie z kwadratu logicznego przyjmując, że:
    \item Fałszywe jest zdanie przeciwne do zdania: „Każdy kwadrat jest prostokątem”.
    \item Przeprowadź wnioskowanie z kwadratu logicznego, przyjmując, że prawdziwe jest zdanie sprzeczne ze zdaniem: „Żaden pełnomocnik nie jest adwokatem”.
    \item Fałszywe jest zdanie przeciwne do zdania: „Każda umowa sprzedaży jest umową odpłatną”.
    \item Prawdziwe jest zdanie sprzeczne ze zdaniem: „Tylko prokuratorzy są oskarżycielami publicznymi”.
    \item Prawdziwe jest zdanie: „Tylko niektóre akty prawne są ustawami”.
    \item Fałszywe jest zdanie: „Tylko niektóre zbrodnie są czynami zabronionymi”.
\end{enumerate}
\subsection*{Sylogistyka}
Przyjmując pierwszą przesłankę jako większą, wyprowadź wniosek z przesłanek. Jeżeli okaże się to niemożliwe, przyjmij, że druga przesłanka jest większa.
\begin{enumerate}
    \item Niektórzy artyści są rzeźbiarzami. Każdy rzeźbiarz jest twórcą.
    \item Każdy lekarz ma wykształcenie. Żaden lekarz nie jest amatorem.
    \item Niektórzy biegacze nie są maratończykami. Żaden maratończyk nie jest leniwy.
    \item Niektórzy programiści pracują zdalnie. Niektórzy programiści nie znają języka C
    \item Każdy matematyk jest wizjonerem. Niektórzy logicy nie są matematykami. 
\end{enumerate}



\end{document}
